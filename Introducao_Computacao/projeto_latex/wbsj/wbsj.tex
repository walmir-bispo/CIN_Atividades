\documentclass[10pt]{paper}
\usepackage[utf8]{inputenc}

\title{Sistemas Inteligentes}
\author{Walmir Bispo }
\date{Novembro 2019}

\usepackage{natbib}
\usepackage{graphicx}


\usepackage{indentfirst}
\setlength{\parskip}{0.2cm}
\begin{document}


\section{Introdução}


\begin{figure}[!htb]
\centering
\includegraphics[scale = 0.7]{SI.jpg}
\caption{Sistemas Inteligentes}
\label{Rotulo}
\end{figure}

Podemos definir Sistemas Inteligentes como um campo da computação que se refere as diversas aplicações da inteligência artificial (IA). A disciplina de Sistemas Inteligentes (SI) é ofertada tanto para os cursos de Ciência quanto Engenharia da computação, e é dividida em duas partes. Cada parte é lecionada por um professor diferente e de forma independente, sendo a parte I, geralmente, um aprofundamento em algoritmos de busca e em sistemas baseados em conhecimento.

Na computação, um algoritmo de busca é um código projetado para encontrar um determinado elemento com propriedades específicas em um conjunto desses elementos. Na disciplina, os algoritmos de busca são divididos em 3 ou 4 módulos, onde em cada módulo o discente se debruça sobre diferentes técnicas de busca e resolução de problemas. Após esta fase, inicia-se os estudos relacionados aos sistemas baseados em conhecimento, que é basicamente uma subárea da inteligência artificial que lida com programas de computador que se utilizam de conhecimento e informação de maneira inteligente para encontrar soluções eficientes. \cite{cinwiki_SI}

Por fim, durante a parte II da disciplina o foco principal gira em torno dos conceitos de inteligência artificial, aprendizagem de máquina, redes neurais, árvores de decisão e lógica difusa. Logo após isto, o estudante terá bases sólidas para arquitetar sistemas que possam chegar a conclusões inteligentes e analisar padrões, em outras palavras, aprender.

\section{Relevância}
Atualmente, graças aos avanços da IA, os sistemas inteligentes possibilitam que máquinas aprendam, processem ou se adaptem a novos dados e realizem tarefas cada vez mais complexas e com maior precisão. Um exemplo bastante interessante é o Google Translate, que possuía milhões de linhas de código, contudo, agora não passam de algumas centenas de linhas de machine learning e IA. Pode-se citar também o IBM Watson, um dos sistemas inteligentes mais famoso no mercado.

Estes são um dos principais motivos pelos quais esta área da computação ganhou tanto destaque nas últimas décadas, além de tudo pode-se destacar:


\begin{itemize}
    \item SIs se adaptam de forma contínua. A IA encontra padrões nos dados para que o algoritmo seja capaz de “ensinar a si mesmo” como resolver deteminado problema.\cite{wikipedia_2019}
    
    \item SIs podem aprimorar um certo produto, a título de exemplo, temos a Siri que foi adicionada aos produtos da Apple.
\end{itemize}

SIs também são aplicáveis em áreas importantes, como na saúde (Auxiliam nas leituras de raio X), no mundo esportivo (Usada para analisar partidas e fornecer aos treinadores maneiras sobre como organizar melhor o time) ou até mesmo na área varejista (Usada na gestão de estoque). A tendência é que não pare por aí e continue obtendo-se avanços cada vez mais significativos, diante disso fica óbvio perceber a necessidade de formar profissionais com conhecimento de controlar e manipular SIs, não é à toa que se trata de uma disciplina obrigatória no perfil curricular dos estudantes do CIn.\cite{michels_2018}\cite{sas}

\section{Relação com outras disciplinas}
A disciplina de Sistemas Inteligentes tem íntima relação com a inteligência Artificial, pois praticamente todo sistema inteligente se apoia sob a IA para funcionar e construir sua base de conhecimento. Já as redes neurais (IF702) e a aprendizagem de máquina (IF699) auxiliam o sistema na busca profunda dos dados, além da sua análise e processamento. Lógica para computação (IF673) e Algoritmos e estrutura de dados (IF672) constituem toda a estrutura dos SIs, por este motivo são pré-requisitos obrigatórios antes de cursar a cadeira.

\bibliographystyle{plain}
\bibliography{wbsj}

\end{document}
